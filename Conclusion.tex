\chapter{Conclusion and Future Work}

This project consists of two main contributions to further research on this topic. First, the Cloud Detection Framework, which simplifies the process of training and testing of deep learning-based detection algorithms on different data. A considerable amount of the work done in this project has been spent implementing the Cloud Detection Framework. To verify whether an object detector is sufficiently robust to be used on an autonomous ferry, tests in different environments, weather conditions and locations needs to be done. Also, the data used in training of the object detector greatly affects its performance. Hence, a system which makes the process of training and testing more efficient may be of great value in the continued research on this topic.

\vspace{1mm}
\noindent
The other main contribution was the collection and labeling of a relatively large dataset, and training and testing on this data. The research done in this project gives an indication of where continued research should be focused, and sets a benchmark for new implementations. That way training on new data, and implementations of new models can easily be compared to previous models, and lets the user know if the new model was a step in the right direction or not. The results from this project shows that both Yolo and SSD has potential to detect boats robustly. There are some situations where the models does not perform ideally, but these cases are in most situations connected to the training data. The detection models respond well to training, and with a sufficiently diverse training dataset they have potential to contribute to an improved understanding of the surroundings of an autonomous ferry. However, this is something that needs to be investigated further. The limitations of the object detector should be discovered further, to better understand its performance in different scenarios. How well the results are in foggy weather conditions, at night or how close boats has to be before they are detected should be investigated further. 

\vspace{1mm}
\noindent
As mentioned in chapter \ref{sec:obj_det}, there has not been done much research on the limitations when it comes to deep learning-based object detectors in maritime environments. Neither has there been done extensive research on how different training data affects the results in different maritime environments. In this project a starting point for this has been done. While that datasets in this report does not include datasets in different weather and lighting conditions, it sets up a system which makes this easy to test in the future. It also provides trained models with performance statistics, which can serve as a benchmark for new implementations.

\section{Future work}
\subsection{Test on video, fps}
An important aspect that has not been tested in this paper is the processing rate of the algorithms on live video. While both Yolo and SSD states processing times in their papers, this should be verified on local hardware. However, since both papers report approximately the same proccessing rate (between 20-150 FPS depending on version of SSD and Yolo), it is more trustworthy than if they had reportes differently.


\subsection{More data}
The most important matter in the future work on this project is to gather more data and conduct tests that continues to analyze Yolo, SSD and other detections algorithms potential. The only way to find the limitations of these models are to test and train them on new data to find what makes them better, and what makes them worse. The data should be gathered in relevant environments and be as diverse as possible, containing different weather, boats and locations.


\subsection{Tracking}
There has not been implemented a tracking algorithm in this project. A tracking algorithm could be of used in post processing of the detections and could help create a more robust understanding of the situation. A tracking algorithm could counter the effect of lost detections for few frames, as discussed in \ref{sec:vid_kams}.

